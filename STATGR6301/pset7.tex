\documentclass[english]{article}
\usepackage[margin=1in]{geometry} 
\usepackage{amsmath,amsthm,amssymb,amsfonts}
\usepackage{comment}
\usepackage{dsfont}
 \usepackage[utf8]{inputenc}
\usepackage[T1]{fontenc}
\usepackage{babel}
\usepackage{graphicx}
\usepackage{tikz,tkz-tab}
\usepackage{framed}
\newcommand{\ve}{\varepsilon}
\newcommand{\N}{\mathbb{N}}
\newcommand{\Z}{\mathbb{Z}}
\newcommand{\R}{\mathbb{R}}
\newcommand{\C}{\mathbb{C}}
\newcommand{\E}{\mathbb{E}}
\newcommand{\Q}{\mathbb{Q}}
\newcommand{\K}{\mathbb{K}}
\newcommand{\RX}{\mathbb{R}[X]}
\newcommand{\KX}{K[X]}
\newcommand{\MatC}{\mathcal{M}_n(\mathbb{C})}
\newcommand{\MatR}{\mathcal{M}_n(\mathbb{R})}
\newcommand{\I}{\mathds{1}}
\newcommand{\df}{\mathrm{d}}
\newcommand{\A}{\mathcal{A}}
\newcommand{\F}{\mathcal{F}}
\newcommand{\J}{\mathcal{J}}
\newcommand{\Cl}{\mathcal{C}}
\usepackage{fancyhdr}
\pagestyle{fancy}
\fancyhf{Assignment 7}
\fancyhead[LE,RO]{}
\fancyhead[RE,LO]{Claire He}
\fancyfoot[CE,CO]{\leftmark}
\fancyfoot[LE,RO]{\thepage}
\usepackage{layouts}

\makeatletter
\renewcommand*{\@textcolor}[3]{%
  \protect\leavevmode
  \begingroup
    \color#1{#2}#3%
  \endgroup
}
\makeatother

\newenvironment{changemargin}[2]{\begin{list}{}{%
\setlength{\topsep}{0pt}%
\setlength{\leftmargin}{0pt}%
\setlength{\rightmargin}{0pt}%
\setlength{\listparindent}{\parindent}%
\setlength{\itemindent}{\parindent}%
\setlength{\parsep}{0pt plus 1pt}%
\addtolength{\leftmargin}{#1}%
\addtolength{\rightmargin}{#2}%
}\item }{\end{list}}
%\begin{changemargin}{2cm}{-1cm}
%Ceci permet d'augmenter la marge gauche de 2cm et de diminuer celle de droite de 1cm.
%\end{changemargin}
\begin{document}
 
%\renewcommand{\qedsymbol}{\filledbox}
%Good resources for looking up how to do stuff:
%Binary operators: http://www.access2science.com/latex/Binary.html
%General help: http://en.wikibooks.org/wiki/LaTeX/Mathematics
%Or just google stuff
 
\title{Probability Theory I \\ 
Assignment 7}
\author{Claire He}
\maketitle

%%%%%%%%%%%%%%%%%%%%%%%



%linewidth: \printinunitsof{cm}\prntlen{\linewidth}

%%%%%%%%%%%%%%%%%%%%%%%%%%%%%%%%%%%%%%%%%%%%%%%%%%%%%%%%%%%%%%%%%%%%%%%%%%%%%%%%%%%%%%%%%%%%%%
\section*{Exercise 1}

Let $\Cl$ be a monotone class and a field. Since it is a field, it is closed under complementation and finite union or intersections. Since it is monotone class, it is closed under increasing countable unions and countably decreasing intersections.

\begin{itemize}
    \item $\varnothing \in \Cl$ since $\Cl$ is a field. 
    \item $\Cl$ is closed under complementation from $\Cl$ being a field.
    \item Now let's take $(A_n) \in \Cl^\N$. Define $B_n = \bigcup_{k=1}^n A_k$ for  $n \in \N$. Since $\Cl$ is a field, $B_n \in \Cl$ as finite union of elements of $\Cl$, and moreover $(B_n)$ defines an increasing sequence. Since $\Cl$ is a monotone class, $\bigcup_{n =1}^{\infty} B_n \in \Cl$. Now this writes also  $\bigcup_{n =1}^{\infty} B_n = \bigcup_{n =1}^{\infty} \bigcup_{k=1}^n A_k = \bigcup_{k =1}^{\infty} A_k  \in \Cl$ as required for a $\sigma$-field. 
    
\end{itemize}



\section*{Exercise 2}

Now let $\Cl$ be a $\pi$-system and a $\lambda$-system. Since it is a  $\pi$-system it is closed under finite intersections and as a  $\lambda$-system , $\Omega \in \Cl$, it is closed under countable increasing unions and monotone differences.

\begin{itemize}
    \item $\varnothing = \Omega \backslash \Omega \in \Cl$ since $\Omega  \in \Cl$
    \item Want to show $\Cl$ is closed under complementation. Take $A\in \Cl$, then $A \subset \Omega$ and $A^c = \Omega \backslash A \in \Cl$ by property of $\lambda$-systems. 
    \item Now let's take $(A_n) \in \Cl^\N$. By the previous assertion, note that for $n$, $A_n^c \in \Cl$. Define $B_n = \bigcup_{k=1}^n A_k$. Since $\Cl$ is a $\lambda$-system, $B_n^c = \bigcap_{k=1}^n A_k^c \in \Cl$ as finite intersections of elements of $\Cl$. By complementation again, $B_n \in \Cl$.  As before, $(B_n)$ defines an increasing sequence. Since $\Cl$ is closed under countable increasing unions, $\bigcup_{n =1}^{\infty} B_n \in \Cl$. Now this writes also  $\bigcup_{n =1}^{\infty} B_n = \bigcup_{n =1}^{\infty} \bigcup_{k=1}^n A_k = \bigcup_{k =1}^{\infty} A_k  \in \Cl$ as required for a $\sigma$-field. 
\end{itemize}

\section*{Exercise 3}
Suppose $\mathcal{I}, \J \subset \F$ are two $\pi$-systems on $(\Omega, \F, P)$. Following the hint, for $I \in \mathcal{I}$, let's consider $\pi_1 : H \mapsto P(I \cap H)$ and $\pi_2 : H \mapsto P(I)\cdot P(H)$ measures on $\J$. 
We need to check they are indeed measures. 
\begin{itemize}
    \item $\pi_1(\varnothing) = P(\varnothing \cap I) = P(\varnothing) = 0$ and $\pi_2(\varnothing) = P(\varnothing) \cdot P( I)  = 0$ 
    \item Let $A_1,  A_2, \dots \in \J$ pairwise disjoint,  

	\begin{eqnarray*}
		\pi_1(\bigcup_{n \in \N} A_n) &  = & P(I \cap \bigcup_{n \in \N} A_n) \\
		& = & P(\bigcup_{n \in \N} A_n \cap I ) \\
	& = & \sum_{n \in \N} P(A_n \cap I) \text{ \ since P is a measure} \\
 & = & \sum_{n \in \N} \pi_1(A_n) 
	\end{eqnarray*} 
Similarly, 
	\begin{eqnarray*}
		\pi_2(\bigcup_{n \in \N} A_n) &  = & P(I) \cdot P(\bigcup_{n \in \N} A_n) \\
	& = & P(I) \cdot \sum_{n \in \N} P(A_n) \text{ \ since P is a measure} \\
& = & \sum_{n \in \N} P(A_n) \cdot P(I) \\
 & = & \sum_{n \in \N} \pi_2(A_n) 
	\end{eqnarray*} 
So $\pi_1$ and $\pi_2$ are measures on $\J$.
\end{itemize}

By determination of measures, if $\pi_1 = \pi_2$ on $\J$, since it is a $\pi$-system, then $\pi_1 = \pi_2$ on $\sigma(\J)$.
Indeed, for any $H \in \J$, 
\begin{eqnarray*}
\pi_1(H) & = & P(I \cap H) \text{ \ independence of } \mathcal{I}, \J \\ 
& = & P(I)\cdot P(H) \\
& = & \pi_2(H)
\end{eqnarray*}
In particular, this yields that $\mathcal{I}$ and $\sigma(\J)$ are independent. 

Now fix $H \in \sigma(\J)$ and consider the measures $\tilde{\pi}_1 : I \mapsto P(I \cap H)$ and $\tilde{\pi}_2 : I \mapsto P(I)\cdot P(H)$ on $\mathcal{I}$, we show similarly that the measures coincide on $\mathcal{I}$ hence by determination of measure on $\sigma(\mathcal{I})$. 
Indeed, for any $I \in \mathcal{I}$, 
\begin{eqnarray*}
\tilde{\pi}_1(I) & = & P(I \cap H) \text{ \ independence of } \mathcal{I}, \sigma(\J) \\ 
& = & P(I)\cdot P(H) \\
& = & \tilde{\pi}_2(I)
\end{eqnarray*}
In particular this means that $\sigma(\mathcal{I})$  and $\sigma(\J)$ are independent. 

\section*{Exercise 4}

\end{document}
